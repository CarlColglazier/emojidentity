From online datasets, we have amassed a significant number of Twitter
user-ids. With these ids, we can use the Twitter API to filter down to
some hundreds of users that have emojis in their names or
biographies. From there, we can gather information on some of their
connections and filter them to find connections that also uses an
emoji in their name or biography. The connection will be identified by
type (following, retweeting, Tweet-quoting, responding, or @-ing).

Once a dataset has been amassed of people with emojis we will carry
out analysis on our hypothesis with both statistical and graph
models. From our statistical models we hope to find that around 10%
of overall connections are with users with similar emojis (R1).

As for the graph side, we expect users to form cliques with other
users with similar emojis (R2). We also hypothesize that emojis in
Twitter display names are used to signal belonging in sub-communities
(R3). It furthermore presents a validation method for the theory based
on analysis of the homogeneity of Twitter networks.
