\documentclass[manuscript, authordraft]{acmart}

\begin{document}
\title{Emoji and Identity}

\author{Carl Colglazier}
\affiliation{%
  \institution{North Carolina State University}
  \city{Raleigh}
  \state{NC}
  \country{USA}}
\author{Zackary Allen}
\affiliation{%
  \institution{North Carolina State University}
  \city{Raleigh}
  \state{NC}
  \country{USA}}
\date{\today}
\maketitle

\begin{abstract}
\end{abstract}

\section{Introduction}
Many Twitter users approach the platform as a means of presenting a
formed identity to an imagined audience\cite{marwick2011tweet}. This
activity often has the effect of flattening identity as individuals
try to fit in to larger groups\cite{boyd2008taken}. At the same time,
the amount of metadata associated with Twitter profiles is limited by
design. Twitter has long used ``hovercards'' as a feature which shows
a small set of information about a profile in the
timeline\cite{twitterhover}. Thus when someone encounters an
unfamiliar profile, they can quickly view that user’s display name,
handle, short biography, profile picture, cover photo, number of
followers, and accounts followed. By elevating these short pieces of
metadata, Twitter creates an incentive for users to succinctly
spotlight pieces of their identity to potential audience members
through these attributes.  The main feature of Twitter is the ability
to post and view Tweets. These Tweets are short messages limited to
280 characters and are typically intended for the followers of the
poster, though they often get spread beyond just the user’s followers,
thus encouraging a social community. Tweets can spread in a variety of
ways: they can be retweeted directly, quote retweeted, or shown in
another user’s feed when someone they follow likes the post. Each of
these methods let a user see another user’s Tweet; this is applicable
even if the former does not follow the latter. Thus Twitter users may
be exposed to people outside of their immediate social circles and
make judgements about these accounts on the basis of presented
metadata rather than social association.

Symbols often represent groups and can be political in
nature. Following Benedict Anderson’s concept of ``imagined
communities'' to describe the concept of
nationalism\cite{anderson2006imagined}, for instance, we can think of
national symbols such as flags or seals as materialized
representations of group identity. Indeed, national flags make up a
good part of the set of standardized emoji\cite{unicodeemoji}. These
symbols–even within the context of an international, standardized
system like emoji–can prove controversial. The Information and
Communication Ministry of Indonesia, for instance, sought to remove
emoji depicting same-sex couples in
2016\cite{boellstorff2016against}. The standardization of emoji has
also raised concerns over the extent of their representation and
diversity.

Twitter has supported emoji on their platform since 2014. In fact,
Twitter even developed an open-source emoji library. Beyond the
company itself, Twitter users have also embraced emoji characters and
use them frequently. Emojitracker, a service which logs how Twitter
accounts use emoji, has logged over a two billion instances of the
most popular emoji on the list, the “Face with Tears of Joy”
emoji\cite{emojitracker}. This emoji was declared the 2015 word of the
year by Oxford Dictionaries.

Based on the hypothesis that many Twitter users include emoji in their
profile metadata as a signal for group identity, we predict that users
with certain emojis in their name or biography will connect at a
higher rate with users that have similar emojis. We further expect
some emoji to be associated with this behavior more than others. For
instance, the rose emoji, which is used by members of the democratic
socialist movement, would more likely show homophilic behavior
compared to more generic emoji such as the red heart emoji.

Understanding the social intricacies of emoji use on Twitter can be
valuable for marketing and public reputation. For organizations that
use Twitter, emojis can be used to make their organization seem like
the everyman. If done correctly, those that use these symbols
correctly can pull on the sense of identity that it brings to blend
into the group. By taking advantage of the way emojis can create a
sense of belonging, organizations could use information such as we are
researching to strengthen their social media appearance.


\section{Method}

From online datasets, we have amassed a significant number of Twitter
user-ids. With these ids, we can use the Twitter API to filter down to
some hundreds of users that have emojis in their names or
biographies. From there, we can gather information on some of their
connections and filter them to find connections that also uses an
emoji in their name or biography. The connection will be identified by
type (following, retweeting, Tweet-quoting, responding, or @-ing).

Once a dataset has been amassed of people with emojis we will carry
out analysis on our hypothesis with both statistical and graph
models. From our statistical models we hope to find that around 10\%
of overall connections are with users with similar emojis (R1).

As for the graph side, we expect users to form cliques with other
users with similar emojis (R2). We also hypothesize that emojis in
Twitter display names are used to signal belonging in sub-communities
(R3). It furthermore presents a validation method for the theory based
on analysis of the homogeneity of Twitter networks.

\subsection{Research Questions}

\begin{enumerate}
\item R1: Do users with emojis connect more often with similar emoji users? (computational)
\item R2: Do people tend to form cliques with similar emoji users? What does the overall network look like? (social)
\item R3: Are some emoji more likely to appear in cliques than others? (computational)
\end{enumerate}

\subsection{Challenges}

\begin{itemize}
\item Data scraping and how much data is enough (10k users?) sparsity?
  (computational)
\item Connection representation (computational)
\item Understanding Twitter cliques (social)
\item R3 may change if there aren’t enough cliques
\end{itemize}


\bibliographystyle{abbrv}
\bibliography{main}

\end{document}
